%%
%% This is file `lexample.tex', 
%% Sample file for siam macros for use with LaTeX 2e
%% 
%% October 1, 1995
%%
%% Version 1.0
%% 
%% You are not allowed to change this file. 
%% 
%% You are allowed to distribute this file under the condition that 
%% it is distributed together with all of the files in the siam macro 
%% distribution. These are:
%%
%%  siamltex.cls (main LaTeX macro file for SIAM)
%%  siamltex.sty (includes siamltex.cls for compatibility mode)
%%  siam10.clo   (size option for 10pt papers)
%%  subeqn.clo   (allows equation numbners with lettered subelements)
%%  siam.bst     (bibliographic style file for BibTeX)
%%  docultex.tex (documentation file)
%%  lexample.tex (this file)
%%
%% If you receive only some of these files from someone, complain! 
%% 
%% You are NOT ALLOWED to distribute this file alone. You are NOT 
%% ALLOWED to take money for the distribution or use of either this 
%% file or a changed version, except for a nominal charge for copying 
%% etc. 
%% \CharacterTable
%%  {Upper-case    \A\B\C\D\E\F\G\H\I\J\K\L\M\N\O\P\Q\R\S\T\U\V\W\X\Y\Z
%%   Lower-case    \a\b\c\d\e\f\g\h\i\j\k\l\m\n\o\p\q\r\s\t\u\v\w\x\y\z
%%   Digits        \0\1\2\3\4\5\6\7\8\9
%%   Exclamation   \!     Double quote  \"     Hash (number) \#
%%   Dollar        \$     Percent       \%     Ampersand     \&
%%   Acute accent  \'     Left paren    \(     Right paren   \)
%%   Asterisk      \*     Plus          \+     Comma         \,
%%   Minus         \-     Point         \.     Solidus       \/
%%   Colon         \:     Semicolon     \;     Less than     \<
%%   Equals        \=     Greater than  \>     Question mark \?
%%   Commercial at \@     Left bracket  \[     Backslash     \\
%%   Right bracket \]     Circumflex    \^     Underscore    \_
%%   Grave accent  \`     Left brace    \{     Vertical bar  \|
%%   Right brace   \}     Tilde         \~}


\documentclass[final,leqno]{siamltex}

% definitions used by included articles, reproduced here for 
% educational benefit, and to minimize alterations needed to be made
% in developing this sample file.

\newcommand{\pe}{\psi}
\def\d{\delta} 
\def\ds{\displaystyle} 
\def\e{{\epsilon}} 
\def\eb{\bar{\eta}}  
\def\enorm#1{\|#1\|_2} 
\def\Fp{F^\prime}  
\def\fishpack{{FISHPACK}} 
\def\fortran{{FORTRAN}} 
\def\gmres{{GMRES}} 
\def\gmresm{{\rm GMRES($m$)}} 
\def\Kc{{\cal K}} 
\def\norm#1{\|#1\|} 
\def\wb{{\bar w}} 
\def\zb{{\bar z}} 

% some definitions of bold math italics to make typing easier.
% They are used in the corollary.

\def\bfE{\mbox{\boldmath$E$}}
\def\bfG{\mbox{\boldmath$G$}}

\usepackage{amsfonts}

\title{Nondeterministic computation}

% The thanks line in the title should be filled in if there is
% any support acknowledgement for the overall work to be included
% This \thanks is also used for the received by date info, but
% authors are not expected to provide this.

\author{Peifeng Wang\thanks{Guanghua Road 1\#, 34-1-3-5, Yanta District, Xi'an, Shaanxi, P. R. China 710075 ({\tt peifeng\_w@yahoo.com})}}

\begin{document}

\maketitle

\begin{abstract}
The execution of an algorithm on computing machines is studied. By Cantor's diagonal argument, we show that, the number of executions within a computation on a nondeterministic Turing machine is asymptotically greater than the number of steps within a computation on a deterministic Turing machine. Thus a nondeterministic machine can have more computing power than a deterministic machine, therefore suggesting $P\neq NP$.
\end{abstract}

\begin{keywords} 
Nondeterministic
\end{keywords}

\begin{AMS}

\end{AMS}

\pagestyle{myheadings}
\thispagestyle{plain}
\markboth{}{}


\section{Introduction}
Turing machines take strings as input for its computation. The complexity of a computation is measured as a function of the size of the input string. In general, a machine has a finite nonempty set of input alphabet $\Sigma$. $\Sigma^*$ is the set of finite string over $\Sigma$. For a string $\omega\in\Sigma^*$, $|\omega|$ is the length of string $\omega$. For a set X, $|X|$ is its cardinality, a sequence over X is a sequence whose terms are elements of X. Let $\mathbb{N}$ be the set of natural numbers.

\begin{proposition} 
%\label{th:prop}
The length of string $\omega$ is a natural number. i.e. $|\omega|\in\mathbb{N}$
\end{proposition}

Every string $\omega\in\Sigma^*$ can be writen as a sequence of symbols $s_1...s_n$ where $\{s_i\in\Sigma|i\in \mathbb{N} ~ and ~ 1 \le i\le n=|\omega|\}$. Each of these symbols is an alphabet at a specific position within $\omega$. The same alphabet at different positions represent distinct symbols. To encode the alphabet and position of a symbol of $\omega$, a set
$S(\omega)=\{s_iz^i|z\notin\Sigma~ \wedge~ i\in\mathbb{N}~ \wedge~ 1 \le i\le |\omega|~ \wedge~ s_i\in\Sigma~ is~ the~ ith~ symbol~ of~ \omega\}$ can be constructed.

\begin{proposition} 
\label{prop:Card_SW}
The cardinality of $S(\omega)$ equals to the length of string $\omega$. i.e. $|S(\omega)|=|\omega|$
\end{proposition}

Then $S(\omega)$ is a countable set.

For a machine M, a computation over string $\omega\in\Sigma^*$ is equivalent to a computation over the set $S(\omega)$. We study computations associated with sequences over $S(\omega)$.

\section{Computation on a deterministic Turing machine (DTM)}

On a DTM, an algorithm defines an order in which elements of input set $S(\omega)$ are read and computed. In each computation, the algorithm reads the elements of input set $S(\omega)$ in a single sequence. This sequence of operation is called an execution. One computation on a DTM can have a single execution..

\begin{proposition} 
\label{prop:DTM_Countable}
Deterministic Turing machine can only compute over a countable input set.
\end{proposition}

\begin{proof}
A deterministic Turing machine executes in a step-by-step manner, it can be shown that each of the steps is associated to a unique natural number.

A deterministic Turing machine is a tuple $(\Sigma,\Gamma,Q,\delta)$, where Q is nonempty finite set of states containing $q_0,q_{accept},q_{reject}$, $\Gamma$ is nonempty finite set of tape alphabet, $\delta$ is the transition function

$\delta:(Q-\{q_{accept},q_{reject}\})\times\Gamma->Q\times\Gamma\times\{1,-1\}$

With the internal state $q\in Q$ and scanned symbol $s\in\Gamma$ as input, $\delta$ defines the next state and symbol to be scanned.

We assign natural numbers to the computation states:

1)The initial state $q_0$ is associated with natural number 0.

2)if $q^-\neq q_{accept}$ and $q^-\neq q_{reject}$ is associated with natural number i, then $q^+$ of $(q^+,s',h)=\delta(q^-,s)$ is associated with number i+1.

Then with mathematical induction, the state of each step in an execution can be associated with a unique natural numbers. Thus an execution can have at most countable steps, and can compute only over countable input set.

\end{proof}

Essentially, the deterministic transition function enables the application of mathemetical induction to prove theorem \ref{prop:DTM_Countable}. Such property is absent on a nondeterministic machine.

\section{Computation on a nondeterministic Turing machine (NDTM)}
On a nondeterministic Turing machine(NDTM) $M_N$, multiple transitions are allowed at any "moment``, and multiple sequences of transitions are executed. Each of the sequences of transitions is called an execution. A computation can have multiple executions.

\begin{proposition} 
\label{prop:NDTM_uncountable}
In a computation on a NDTM, every sequence over the input set $S(\omega)$ has a distinct execution.
\end{proposition}

\begin{proof}
We track all of the executions by recording the sequence of input symbols read by the NDTM.

For each execution e, there is an order in which the machine $M_N$ reads the symbols of input set $S(\omega)$. Initially, execution e sets its record R(e) to an empty string, when execution e scans an input symbol $\alpha\in S(\omega)$, the symbol is appended to the existing R(e). Thus the records R of all executions are sequences over the input set $S(\omega)$, the BNF grammar of the recorded sequence is

$\begin{array}{ccc}
R&::=&R\alpha|\alpha\\
\alpha&::=&elements~ of~ S(\omega)
\end{array}
$

Then R includes all strings whose alphabets are elements of $S(\omega)$. Every string $r\in R$ is a record of an execution e, then e is the corresponding execution of r.

For 2 exeuctions $e_1$ and $e_2$, if $e_1= e_2$, then they have the same record string $R(e_1)= R(e_2)$. Equivalently, if $R(e_1)\neq R(e_2)$, $e_1\neq e_2$.

\end{proof}

A computation on a NDTM contains the set of all executions.

\section{$P\neq NP$}

To put it simply, the number of executions within a computation on a NDTM is asymptotically greater than the number of steps within a computation on a DTM.

In general, for fixed $k\in\mathbb{N}$, the kth Cartesian power of a set X is $X^k=\{(x_1,...,x_k)|~ x_i\in X~ for~ all~ 1\le i\le k\}$, if X is a finite set of cardinality $|X|$, the cardinality of its Cartesian power is $|X^k|=|X|^k$. If X is a countably infinite set and k is finite, its Cartesian power $X^k$ is still a countable set.

On a computing machine with input $\omega$, an execution within $|\omega|^k=|S(\omega)|^k$ steps can only compute the kth Cartesian power of the input set $S(\omega)$, i.e. $(S(\omega))^k$. Let P(n) be a polynomial of finite degree, there exists finite $k\in\mathbb{N}$ such that $|\omega|^k$ is asymptotically greater than $P(|\omega|)$

$\lim_{|\omega|\rightarrow\infty}|\omega|^k > \lim_{|\omega|\rightarrow\infty}P(|\omega|)$

Let $E(\omega)$ be the set of all executions of a computation on a NDTM with input $\omega$. If the input string $\omega$ is infinitely long, we have 1) from proposition \ref{prop:Card_SW}, $S(\omega)$ is countably infinite. 2) from Cantor's diagonal argument, the set of all sequences over $S(\omega)$ is uncountable. 3) from theorem \ref{prop:NDTM_uncountable}, the size of $E(\omega)$ is greater than the size of the set of all sequences over $S(\omega)$, thus $E(\omega)$ is uncountable. 4) any finite Cartesian power of $S(\omega)$ is countable. Thus for any finite k,

$\lim_{|\omega|\rightarrow \infty}|E(\omega)|>\lim_{|\omega|\rightarrow \infty}|S(\omega)|^k=\lim_{|\omega|\rightarrow \infty}|\omega|^k$

then for any polynomial P(n) of finite degree

$\lim_{|\omega|\rightarrow \infty}|E(\omega)|>\lim_{|\omega|\rightarrow \infty}P(|\omega|)$

On the other hand, from theorem \ref{prop:DTM_Countable}, a computation on a DTM can have only countable steps.

Thus in the case of infinitely long input string, the number of executions on a NDTM is strictly greater than the number of steps on a DTM, thus nondeterministic machine can have more computing power than a deterministic machine.

The above case of infinitely long input string also applies when the input set $S(\omega)$ is enough large. In specific, for any polynomial $P(n)$ of finite degree, there exists m, k, for all $|S(\omega)|>m$, 1) the number of steps that a DTM can compute within polynomial time $P(|\omega|)$ is less than $|\omega|^k$ for some k. 2)  the size of kth Cartesian power of $S(\omega)$ is strictly less than the number of all executions of a computation on a NDTM. Thus $P<NP$

\section*{Acknowledgments}
 
 
\begin{thebibliography}{10} 
\bibitem{Turing} {\sc Barker-Plummer, David}, 
{\em Turing Machines}, The Stanford Encyclopedia of Philosophy (Summer 2013 Edition), Edward N. Zalta (ed.), URL = http://plato.stanford.edu/archives/sum2013/entries/turing-machine/
 
\end{thebibliography} 

\end{document} 

